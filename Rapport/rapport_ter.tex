

\documentclass{article}
\usepackage[utf8]{inputenc}

 
\begin{document}

\begin{center}
{\large Master 1 - Ingénierie Informatique} \\ [0.5cm]
\vfill
\rule{\linewidth}{0.4mm} \\ [0.4cm]
{\huge \bfseries
Travaux d'Etude et de Recherche\\
- \\
GraphoScan \\ [0.4cm]
Mémoire Intermédiaire \\ [0.4cm]
}
\rule{\linewidth}{0.4mm} \\ [1.5cm]

\begin{minipage}{0.4\textwidth}
\begin{flushleft} \large
Thibault \textsc{Charpignon} \\
Benoît \textsc{Gallet} \\
Emmanuel \textsc{Herrmann} \\
Martin \textsc{Rety}
\end{flushleft}
\end{minipage}

\vfill

\large\emph{Encadré par : }{Matthieu \textsc{Exbrayat}}

\vfill


{\large 6 Février - 13 Mars 2017}

\end{center}

\newpage
 
\tableofcontents

\newpage
 
\section{Résumé du projet}

\subsection{Présentation}

Ce projet de TER prolonge un travail déjà entamé l'année dernière par deux étudiants de Polytech

\begin{itemize}
\item Présentation rapide du projet : Présenter brièvement le sujet du TER
\item Les origines, la demande du paléographe : Dans quel but fait-on ça (avoir une reproduction 3D des gestes de la main pour distinguer des écoles d'écritures, ...)
\end{itemize}
\section{Introduction au domaine}
\begin{itemize}
\item Librairies : utilisation librairies graphique pour du traitement de flux vidéo dans le but de faire du tracking sur le résultat
\item Langages,Frameworks : opencv matlab c++
\end{itemize}
\section{Analyse de l'existant}
\begin{itemize}
\item Résumé du travail des deux précédents étudiants
\item Pour l'instant une capture limitée à 10fps
\item Impossibilité de bouger la feuille
\item D'autres petits problèmes : l'éclairage, les mouvements parasites du calligraphe, etc
\item Les fonctionalités déjà implémentées : La synchronisation des caméras, le calibrage, les algorithmes de tracking basiques, une première recherche et analyse du matériel
\end{itemize}
\section{Besoins non fonctionnels et fonctionnels}
-Benoit + Emmanuel (Augmentation de la cadence d'acquisition):
\begin{itemize}
\item Programmation parallèle (cpu ou gpu)
\item Analyse de la complexité de l'algorithme et des structures de données
\item Faire des tests fonctionnels dans l'algo pour tester les performances
\item Généraliser le problème pour fonctionner avec plus de deux caméras
\end{itemize}

-Thibault + Martin (Tracking de la plume):
\begin{itemize}
\item Analyse de la complexité des algos de tracking
\item Eliminer les mouvements parasites
\item Définir une zone de capture
\item Gérer les sorties du champs de vision des caméras
\end{itemize}

-Général
\begin{itemize}
\item Idées de rajouter une caméra pour ajouter un flux vidéo de suivi de la plume
\item Améliorations diverses pour la lumière (comme un panneau LED)
\item Compatibilité Windows - Linux - Mac
\end{itemize}

\section{Prototypes et résultats de tests préparatoires}
A définir plus tard

\section{Planning}
Deux sous-groupes
Martin et Thibault sur le tracking et ce qui tourne autour, Benoit et Emmanuel les caméras et l'amélioration de l'acquisition stéréo
Ajouter diagramme de gantt actuel + prévisionnel (?)

\section{Bibliographie}
A remplir au fur et à mesure de l'exploration du projet (librairies fournies et sites de recherche spécialisés) + Noter les sites visités dans un fichier partagé
 
 
\end{document}

\grid
