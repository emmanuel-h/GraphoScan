\documentclass{article}
\usepackage[utf8]{inputenc}
\usepackage{indentfirst}
 
\begin{document}

\begin{center}
{\large Master 1 - Ingénierie Informatique} \\ [0.5cm]
\vfill
\rule{\linewidth}{0.4mm} \\ [0.4cm]
{\huge \bfseries
Travaux d'Etude et de Recherche\\
- \\
GraphoScan \\ [0.4cm]
Mémoire Intermédiaire \\ [0.4cm]
}
\rule{\linewidth}{0.4mm} \\ [1.5cm]

\begin{minipage}{0.4\textwidth}
\begin{flushleft} \large
Thibault \textsc{Charpignon} \\
Benoît \textsc{Gallet} \\
Emmanuel \textsc{Herrmann} \\
Martin \textsc{Réty}
\end{flushleft}
\end{minipage}

\vfill

\large\emph{Encadré par : }{Matthieu \textsc{Exbrayat}}

\vfill


{\large 6 Février - 13 Mars 2017}

\end{center}

\newpage
 
\tableofcontents

\newpage
 
\section{Résumé du projet}

\subsection{Présentation}

Ce projet de TER prolonge un travail déjà entamé l'année dernière par deux étudiants de Polytech, consistant à enregistrer en vidéo l'écriture d'un calligraphe, pour pouvoir reconstruire un modèle en 3D du mouvement de la plume. Une structure en bois supporte deux caméras, que l'on peut bouger le long de rails puis fixer à l'aide de vis. Le calligraphe écrit sous cette structure et la feuille est éclairée par des spots lumineux. Il faut alors associer l'image des deux caméras, ce qui n'est pas possible nativement avec le logiciel fourni par le fabricant (FlyCapture de PointGrey) pour faire de l'acquisition vidéo en stéréo, puis reconstituer via OpenGL les mouvements de la plume. Ces mouvements ont été sauvegardés grâce à des algorithmes de tracking, travaillant sur les vidéos enregistrées auparavant.

\subsection{But du projet}

Ce projet permettra à terme de réaliser une reconstitution 3D des mouvements du calligraphe. On pourra alors lui faire recopier plusieurs textes, provenant de différents lieux et différentes époques, afin de pouvoir comparer les styles d'écriture, définir s'il existait différentes écoles d'écriture, différents styles, etc. De manière plus générale, le projet pourra servir pour beaucoup d'applications par la suite, car le code final se voudra le plus généraliste possible.

\begin{itemize}
\item Présentation rapide du projet : Présenter brièvement le sujet du TER
\item Les origines, la demande du paléographe : Dans quel but fait-on ça (avoir une reproduction 3D des gestes de la main pour distinguer des écoles d'écritures, ...)
\end{itemize}
\section{Introduction au domaine}
\begin{itemize}
\item Librairies : utilisation librairies graphique pour du traitement de flux vidéo dans le but de faire du tracking sur le résultat
\item Langages,Frameworks : opencv matlab c++
\end{itemize}

\section{Analyse de l'existant}

\section{Analyse de l'existant}

\subsection{Fonctionnalités déjà implémentées}

Le programme tel qu'il nous a été fourni dispose de plusieurs fonctionnalités élémentaires à son bon fonctionnement. Parmi ceci, nous pouvons compter :
\begin{itemize}
\item Le calibrage du dispositif dans sa globalité (caméras + surface d'écriture)
\item La synchronisation des deux caméras pour une reconstitution en trois dimensions des gestes lors de l'écriture
\end{itemize}

\subsection{Fréquence d'acquisition limitée}

Jusqu'alors, le programme ne possède pas une fréquence d'acquisition suffisante de l'image. En effet, cette dernière n'est de l'ordre que de huit images par seconde. De ce fait, cette valeur ne permet pas une reconstituion précise en trois dimensions des gestes du calligraphe. L'objectif ici est donc de se rapprocher le plus possible de la fréquence maximale d'acquisition des caméras du dispositif, soit trente images par seconde et ainsi gagner en précision lors du traitement.

\subsection{Impossibilité de bouger la feuille}

Avec la configuration actuelle, le calligraphe a l'impossibilité de bouger la feuille sur laquelle il écrit sous peine de perdre les réglages définis auparavant. Instinctivement, la personne qui écrit peut souhaiter bouger cette feuille et ainsi gagner en confort. Le but serait donc de trouver un moyen de gérer un changement de position de la feuille sans que cela n'affecte les résultats, en recalculant les réglages par exemple.

\subsection{Problèmes divers}

Initialement, l'éclairage du dispositif se faisait à l'aide de deux spots disposés de part et d'autre du calligraphe. Le problème majeur d'un tel moyen est la présence d'ombres à certains endroits rendant le traitement des images difficile, ainsi que la chauffe des lampes pouvant se réveler gênant à la longue. Un autre problème à gérer sont les gestes "inutiles" à l'acquisition que le calligraphe peut faire. En effet ce dernier peut par exemple vouloir étendre son bras pour se relaxer, geste qui sera pris en compte par le programme dans la reconstitution.

\section{Besoins non fonctionnels et fonctionnels}

	Une séparation des tâches était déjà effective dans le projet original, sur lequel travaillaient deux personnes : une personne travaillait sur l'acquisition stéréo, pendant que la seconde s'occupait de la modélisation 3D de la plume. Cette séparation a été conservée dans ce TER : \textsc{Gallet} Benoît et \textsc{Herrmann} Emmanuel s'occupent de la première partie, tandis que \textsc{Charpignon} Thibault et \textsc{Réty} Martin ont pris la deuxième.

\subsection{Augmentation de la cadence d'acquisition}

Naturellement, différentes idées nous sont venues pour augmenter la cadence d'acquisition de la vidéo en stéréo. Nous les détaillons ici, même si par la suite cette liste sera sûrement étoffée. Le programme fonctionne suivant plusieurs étapes : Tout d'abord, une image est capturée à partir des deux caméras, puis les deux images sont encodées dans une vidéo. Une fois la capture finie, la vidéo est traitée afin de régler les problèmes de distorsion. Grâce à cette grande boucle qui capture les images deux par deux (une image par caméra), les vidéos finales commencent et terminent exactement au même moment, et permettent donc d'avoir exactement au même moment la feuille d'écriture filmée sous deux angles différents. La modélisation en 3D sous OpenGl est alors possible. La seule variable est que plus la cadence d'acquisition est élevée, plus il y aura de fps sur les vidéos finales, et plus la modélisation 3D de la plume sera précise. De plus, notre architecture et notre code doivent être assez robuste, pour que si un jour une troisième voire une quatrième caméra soient rajoutées, le nombre de fps ne redescende pas drastiquement.

\subsubsection{Programmation parallèle}

Grâce à la programmation parallèle, qu'elle soit au niveau du CPU avec de l'openmp ou des threads, ou au niveau du GPU avec CUDA, nous pensons pouvoir accélérer l'acquisition des images, et donc des fps sur les vidéos finales. Nous pensons regarder quelles parties peuvent être faites en parallèle, peut-être est-il possible d'uniquement récupérer une image tous les 3 centièmes de secondes (pour les 30 fps) dans le programme principal, et de faire tous les autres traitements dans des régions parallèles, avec par exemple un thread qui s'occupe d'ajouter la prochaine image à la vidéo, un autre thread qui enlève la distorsion de l'image, etc.

\subsubsection{Complexité}

Reprendre le code pour en examiner sa complexité est une autre piste envisagée pour augmenter les fps. Nous pensons séparer cette idée en deux étapes : Tout d'abord regarder la complexité de l'algorithme dans sa généralité, pour se rendre compte s'il y a problème ou non à ce niveau là, et voir les morceaux posant plus problème que le reste, puis faire des tests plus précisément sur ces parties pour voir précisément ce qui ne va pas. Une analyse légerement différente pourra être effectuée, avec des tests fontionnels calculant quelle partie prend le plus de temps. Ces tests sont très complémentaires des tests de complexité, à eux deux ils devraient mettre en exergue les problèmes principaux du code actuel.

\subsubsection{Modularité}

Outre cette analyse de la complexité, une mise au propre du code devra être effectuée. En effet, tout se trouve dans la fonction main, dans deux grandes boucles. Une partie de notre travail sera donc de modulariser cette fonction, de la séparer en plusieurs méthodes afin de gagner en clarté. La complexité ne devrait pas être touchée car nous rajoutons l'option \texttt{-O3} lors de la compilation afin d'avoir un code en une ligne. Ce nettoyage permettra de naviguer plus facilement dans le code, et de faciliter sa maintenance par la suite.

\subsubsection{Généralisation}

Sinon, le dernier axe sur lequel travailler sera la généralisation du nombre de caméra. Pour l'instant, tout dans le code est fait pour deux caméras, avec du code dupliqué deux fois pour chaque action. La généralisation pour n caméras sera facilitée par la modularisation du code, et permettra par la suite de rajouter une ou plusieurs caméras sans modification majeure du code, uniquement en changeant quelques \emph{\#define}.


\subsection{Tracking de la plume}

Comme pour la partie sur l'augmentation de la cadence d'acquisition, différents problèmes sont à résoudre pour le tracking. Cette partie permet de traiter les vid\'eos produites par les cam\'eras et d'en ressortir une trace des mouvements effectu\'es par le calligraphe. L'\'etudiant de Polytech qui a travaill\'e sur cette partie a recherch\'e diff\'erents algorithmes permettant d'effectuer ce tracking. Son \'etude se focalise sur deux algorithmes bas\'es sur l'apprentissage de toutes les apparences observ\'ees de l'objet et d'une estimation des erreurs pour ensuite les \'eviter:

\begin{itemize}

\item Tracking Learning Detection (TLD)

\item Kernelized Correlation Filters (KCF)

\end{itemize}

  
\subsubsection{Analyse de la complexité}

Heureusement, l'analyse des deux algorithmes a déjà été faite par l'\'etudiant, ce qui a montr\'e que dans notre cas l'algorithme KCF est le plus efficace. Son \'etude est bas\'ee sur plusieurs crit\`eres, la d\'eviation moyenne des deux vid\'eos, le nombre de frames et le temps de calcul. Seul le premier critère est r\'eellement diff\'erent entre les deux m\'ethodes. C'est cette diff\'erence qui a orient\'e son choix vers l'algortihme KCF. \\

Ici, notre premier axe de recherche sera orient\'e vers une \'etude compl\'ementaire de ces algorithmes pour v\'erifier la v\'eracit\'e de l'analyse pr\'ec\'edente. Pour cela nous allons r\'eutiliser les crit\`eres d'\'etudes et ensuite essayer d'en trouver d'autres pour confirmer le choix. Dans un second temps il nous faudra rechercher d'autres algorithmes ou m\'ethodes de programmation pour am\'eliorer le tracking.

\subsubsection{Gestion mouvements}

Bien entendu, le choix des algorithmes n'est pas la seule difficulté, nous faisons face \'egalement à des contraintes physiques li\'ees aux mouvements du calligraphe. Par exemple il doit prendre des temps de repos afin de garder sa fluidit\'e d'\'ecriture en faisant des gestes de relaxation du poignet. Ces mouvements ne doivent pas \^etre pris en compte par l'algorithme de tracking afin d'\'eviter des erreurs sur la repr\'esentation du mouvement. \\

Résoudre ce problème ce probl\`eme pourrait passer par la sauvegarde \`a un temps T et \`a un temps T+1 d'une image de la partie suivie. Puis analyser la diff\'erence entre les deux images et en ressortir un r\'esultat positif ou n\'egatif. Cela reviens \`a prendre la derni\`ere image o\`u le calligraphe \'ecrit et une autre image qui permettra de voir si le mouvement est la continuit\'e de l'\'ecriture ou un mouvement parasite.

-Général
\begin{itemize}
\item Idées de rajouter une caméra pour ajouter un flux vidéo de suivi de la plume
\item Améliorations diverses pour la lumière (comme un panneau LED)
\item Compatibilité Windows - Linux - Mac
\item Zone de capture
\end{itemize}

\section{Prototypes et résultats de tests préparatoires}
A définir plus tard

\section{Planning}
Deux sous-groupes
Martin et Thibault sur le tracking et ce qui tourne autour, Benoit et Emmanuel les caméras et l'amélioration de l'acquisition stéréo
Ajouter diagramme de gantt actuel + prévisionnel (?)

\bibliographystyle{plain}
\nocite{*}
\bibliography{Modules/biblio}
 
 
\end{document}

\grid
