
Le programme en lui-même est entièrement réalisé en C++, et différentes bibliothèques graphiques sont utilisées dans ce projet pour le traitement du flux vidéo, dans le but de faire du tracking sur le résultat, notamment OpenGL et OpenCV. OpenGL est utilisé pour la reconstruction du mouvement de la plume en 3D, tandis qu'OpenCV est plus utilisé pour le traitement de l'image, notamment toutes les opérations faites dessus. De plus, Matlab est utilisé pour effectuer diverses opérations mathématiques sur les images, comme par exemple pour la calibration originelle servant à l'alignement pour la stéréo, ainsi que pour l'enlèvement de la distorsion. En effet, comme les caméras ont un grand angle, les éléments sur le bord de l'image sont courbés, il faut donc les remettre droits pour pouvoir travailler dessus.