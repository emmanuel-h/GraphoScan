\subsection{Fonctionnalités déjà implémentées}

Le programme tel qu'il nous a été fourni dispose de plusieurs fonctionnalités élémentaires permettant son bon fonctionnement. Parmi celles-ci, nous pouvons compter~:
\begin{itemize}
\item Le calibrage du dispositif dans sa globalité (caméras + surface d'écriture)
\item La synchronisation des deux caméras pour une reconstitution en trois dimensions des gestes lors de l'écriture
\end{itemize}
En plus de ces fonctionnalités, existe un dispositif matériel de capture. Ce dernier (Figure \ref{cameras}) est composé d'une structure en bois sur laquelle sont montées deux caméras. Ces dernières sont connectées à l'ordinateur par le biais d'un cable USB 3.0. Il est également possible de les bouger afin d'en ajuster les réglages.

\begin{figure}[!h]
\centering
\includegraphics[width=\textwidth, height=4cm]{Modules/Picture/camerasPic.png}
\caption{Dispositif de capture stéréo}
\label{cameras}
\end{figure}

\subsection{Fréquence d'acquisition limitée}

Jusqu'alors, le programme ne possédait pas une fréquence d'acquisition suffisante de l'image. En effet, cette dernière n'était que de l'ordre de huit images par seconde. De ce fait, cette valeur ne permet pas une reconstituion précise en trois dimensions des gestes du calligraphe. L'objectif ici est donc de se rapprocher le plus possible de la fréquence maximale d'acquisition des caméras, soit trente images par seconde et ainsi gagner en précision lors du traitement.

\subsection{Impossibilité de bouger la feuille}

Avec la configuration actuelle, le calligraphe a l'impossibilité de bouger la feuille sur laquelle il écrit sous peine de perdre les réglages définis auparavant. Instinctivement, la personne qui écrit peut souhaiter bouger cette feuille et ainsi gagner en confort. Le but serait donc de trouver un moyen de gérer un changement de position de la feuille sans que cela n'affecte les résultats, en recalculant les réglages par exemple.

\subsection{Problèmes divers}

Initialement, l'éclairage du dispositif se faisait à l'aide de deux spots disposés de part et d'autre du calligraphe. Le problème majeur d'un tel moyen est la présence d'ombres à certains endroits rendant le traitement des images difficile, ainsi que la chaleur des lampes pouvant se réveler gênante à la longue. Un autre problème à gérer sont les gestes "inutiles" à l'acquisition que le calligraphe peut faire. En effet ce dernier peut par exemple vouloir étendre son bras pour se relaxer, geste qui sera pris en compte par le programme dans la reconstitution.