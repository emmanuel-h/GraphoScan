Comme pour la partie sur l'augmentation de la cadence d'acqui\-si\-tion, dif\-fé\-rents problèmes sont à résoudre pour le tracking. Cette partie permet de traiter les vidéos produites par les caméras et d'en ressortir une trace des mouvements effectués par le calligraphe. L'étudiant de Polytech qui a travaillé sur cette partie a recherché différents algorithmes permettant d'effectuer ce tracking. Son étude se focalise sur deux algorithmes basés sur l'apprentissage de toutes les apparences observées de l'objet, et d'une estimation des erreurs pour ensuite les éviter~:

\begin{itemize}

\item Tracking Learning Detection (TLD)

\item Kernelized Correlation Filters (KCF)

\end{itemize}

  
\subsubsection{Analyse de la complexité}

Heureusement, l'analyse des deux algorithmes a déjà été faite par l'é\-tu\-diant, ce qui a montré que dans notre cas l'al\-go\-ri\-thme KCF est le plus efficace. Son étude est basée sur plusieurs critères, la déviation moyenne des deux vidéos, le nombre de \textit{frames} et le temps de calcul. Seul le premier critère est réellement différent entre les deux méthodes. C'est cette différence qui a orienté son choix vers l'algorithme KCF. \\

Ici, notre premier axe de recherche sera orienté vers une étude complémentaire de ces algorithmes pour vérifier la véracité de l'analyse précédente. Pour cela nous allons réutiliser les critères d'étude et ensuite essayer d'en trouver d'autres pour confirmer le choix. Dans un second temps il nous faudra rechercher d'autres algorithmes ou méthodes de programmation pour améliorer le tracking.

\subsubsection{Gestion mouvements}

Bien entendu, le choix des algorithmes n'est pas la seule difficulté, nous faisons face également à des contraintes physiques liées aux mouvements du calligraphe. Par exemple il doit prendre des temps de repos afin de garder sa fluidité d'écriture en faisant des gestes de relaxation du poignet. Ces mouvements ne doivent pas être pris en compte par l'algorithme de tracking afin d'éviter des erreurs sur la représentation du mouvement. \\

Résoudre ce problème pourrait passer par la sauvegarde à un temps T et à un temps T+1 d'une image de la partie suivie, puis analyser la différence entre les deux images et en ressortir un résultat positif ou négatif. Cela revient à prendre la dernière image où le calligraphe écrit et une autre image qui permettra de voir si le mouvement est la continuité de l'écriture ou un mouvement parasite.