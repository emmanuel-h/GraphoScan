\subsection{Droits d'auteur}
Au fur et à mesure que nous essayions de comprendre le code de la reconstruction 3D (ce qui a été long car il a été fourni sans doc, commentaires et explications d'utilisation, avec une syntaxe et des imports pour Windows), nous nous sommes rendu compte que sur cette partie plusieurs morceaux de code avaient été copiés sur Internet.

La classe Shader, ainsi que la moitié de la classe Camera provient de \url{http://blog.csdn.net/sinat_26989191/article/details/51205149}, un outil pour faire un système solaire en 3D.

La classe HOG est presque entièrement copié collée de la fin de ce programme ( \url{http://lib.csdn.net/snippet/cplusplus/28974} ), qui est sous copyright.

Diverses autres parties du code proviennent de copiés-collés de différentes documentations ou exemples, ce qui est dans une certaine mesure moins dé\-ran\-geant.

Il y a certainement d'autres endroits du projet qui proviennent d'Internet, tout n'a pas été vérifié par manque de temps. Au total, nous estimons qu'au moins 50\% du code n'a pas été écrit par le groupe précédent. Cela implique que si ce projet est repris par un groupe futur, il faudra réécrire ces parties du code, ou du moins indiquer clairement les sources et les différents copyrights sur les classes copiées.

\subsection{Interface graphique}

Une extension possible serait de réaliser une interface graphique, par exemple en Qt. Cela permettrait un maniement beaucoup plus facile de l'application, et rendrait possible son utilisation par des personnes extérieures. En effet, en ce moment tout s'exécute dans le terminal, ce qui peut être moins intuitif. Avec cette interface graphique, on pourrait par exemple choisir de sélectionner une caméra pour voir ce qu'elle enregistre, définir différentes options comme choisir l'algorithme de tracking à utiliser, les sorties vidéos ou images à effectuer, ...