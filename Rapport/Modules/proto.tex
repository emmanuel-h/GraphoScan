Il nous fallait, pour bien prendre en main le projet, tester réellement le dispositif de lancement du logiciel jusqu'à la capture vidéo. Nous avons dû  procéder à l'installation de tout l'environnement de travail nécessaire (FlyCap2, OpenCV, OpenGL) et l’acquisition des premières vidéos avec les caméras mises à notre disposition. \\

Notre première tâche a été de transférer le code initial sous Linux et ainsi le tester directement. Les tests ont été concluants et le premier groupe a pu commencer directement à améliorer le système. Le second, quant à lui, a récupéré le code concernant le tracking mais a rencontré de gros problèmes lors de son passage de Windows vers Linux.
Le code n'utilisant pas des fonctions standards, important des librairies en "dur" et étant peu commenté, il est pour le moment impossible de tester le code de l'étudiant qui travaillait sur le tracking l'année précédente. Il aura fallu passer plusieurs semaines de lecture, compréhension, débuggage, etc pour réussir à faire enfin marcher le tracking et la reconstruction 3D.  Le code n'était pas du tout commenté, il nous a fallu le déchiffrer variable par variable, fonction par fonction, classe par classe. De plus, les shaders OpenGL n'étaient pas présents, il nous a fallu les recréer en comprenant ce qu'il fallait passer à la carte graphique, et ce dont on avait besoin dans le code CPU.