Il nous fallait, pour bien prendre en main le projet, tester réellement le dispositif de lancement du logiciel jusqu'à la capture vidéo. Nous avons dû  procéder à l'installation de tout l'environnement de travail nécessaire (FlyCap2, OpenCV, OpenGL) et l’acquisition des premières vidéos avec les caméras mises à notre disposition. \\

Notre première tâche a été de transférer le code initial sous Linux et ainsi le tester directement. Les tests ont été concluants et le premier groupe a pu commencer directement à améliorer le système. Le second, quant à lui, a récupéré le code concernant le tracking mais a rencontré de gros problèmes lors de son passage de Windows vers Linux.
Le code n'utilisant pas des fonctions standards, important des librairies en "dur" et étant peu commenté, il est pour le moment impossible de tester le code de l'étudiant qui travaillait sur le tracking l'année précédente. Pour y remédier le second groupe a dû repasser cette partie du projet sur Windows le temps de bien comprendre les différents problèmes et de les corriger.