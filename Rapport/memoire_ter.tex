\documentclass{article}
\usepackage[utf8]{inputenc}
\usepackage{indentfirst}
\usepackage[nottoc,notlot,notlof]{tocbibind}
\usepackage{url}
\usepackage{hyperref}
\usepackage{listings}
\usepackage[export]{adjustbox}

\begin{document}
\thispagestyle{empty}
\begin{center}
{\large Master 1 - Ingénierie Informatique} \\ [0.5cm]
\vfill
\rule{\linewidth}{0.4mm} \\ [0.4cm]
{\huge \bfseries
Travaux d'Etude et de Recherche\\
- \\
GraphoScan \\ [0.4cm]
Mémoire Final \\ [0.4cm]
}
\rule{\linewidth}{0.4mm} \\ [1.5cm]

\begin{minipage}{0.4\textwidth}
\begin{flushleft} \large
Thibault \textsc{Charpignon} \\
Benoît \textsc{Gallet} \\
Emmanuel \textsc{Herrmann} \\
Martin \textsc{Réty}
\end{flushleft}
\end{minipage}

\vfill

\large\emph{Encadré par : }{Matthieu \textsc{Exbrayat}}

\vfill


{\large 6 Février - 22 Mai 2017}

\end{center}

\newpage
\setcounter{page}{1}
\tableofcontents

\newpage

\section{Résumé du projet}

\subsection{Présentation}

Ce projet de TER prolonge un travail déjà entamé l'année dernière par deux étudiants de Polytech, consistant à enregistrer en vidéo l'écriture d'un calligraphe, pour pouvoir reconstruire un modèle en 3D du mouvement de la plume. Une structure en bois supporte deux caméras, que l'on peut bouger le long de rails puis fixer à l'aide de vis. Le calligraphe écrit sous cette structure et la feuille est éclairée par des spots lumineux. Il faut alors associer l'image des deux caméras, ce qui n'est pas possible nativement avec le logiciel fourni par le fabricant (FlyCapture de PointGrey) pour faire de l'acquisition vidéo en stéréo, puis reconstituer via OpenGL les mouvements de la plume. Ces mouvements ont été sauvegardés grâce à des algorithmes de tracking, travaillant sur les vidéos enregistrées auparavant.

\subsection{But du projet}

Ce projet permettra à terme de réaliser une reconstitution 3D des mouvements du calligraphe. On pourra alors lui faire recopier plusieurs textes, provenant de différents lieux et différentes époques, afin de pouvoir comparer les styles d'écriture, définir s'il existait différentes écoles d'écriture, différents styles, etc. De manière plus générale, le projet pourra servir pour beaucoup d'applications par la suite, car le code final se voudra le plus généraliste possible.

\section{Analyse de l'existant}

\section{Analyse de l'existant}

\subsection{Fonctionnalités déjà implémentées}

Le programme tel qu'il nous a été fourni dispose de plusieurs fonctionnalités élémentaires à son bon fonctionnement. Parmi ceci, nous pouvons compter :
\begin{itemize}
\item Le calibrage du dispositif dans sa globalité (caméras + surface d'écriture)
\item La synchronisation des deux caméras pour une reconstitution en trois dimensions des gestes lors de l'écriture
\end{itemize}

\subsection{Fréquence d'acquisition limitée}

Jusqu'alors, le programme ne possède pas une fréquence d'acquisition suffisante de l'image. En effet, cette dernière n'est de l'ordre que de huit images par seconde. De ce fait, cette valeur ne permet pas une reconstituion précise en trois dimensions des gestes du calligraphe. L'objectif ici est donc de se rapprocher le plus possible de la fréquence maximale d'acquisition des caméras du dispositif, soit trente images par seconde et ainsi gagner en précision lors du traitement.

\subsection{Impossibilité de bouger la feuille}

Avec la configuration actuelle, le calligraphe a l'impossibilité de bouger la feuille sur laquelle il écrit sous peine de perdre les réglages définis auparavant. Instinctivement, la personne qui écrit peut souhaiter bouger cette feuille et ainsi gagner en confort. Le but serait donc de trouver un moyen de gérer un changement de position de la feuille sans que cela n'affecte les résultats, en recalculant les réglages par exemple.

\subsection{Problèmes divers}

Initialement, l'éclairage du dispositif se faisait à l'aide de deux spots disposés de part et d'autre du calligraphe. Le problème majeur d'un tel moyen est la présence d'ombres à certains endroits rendant le traitement des images difficile, ainsi que la chauffe des lampes pouvant se réveler gênant à la longue. Un autre problème à gérer sont les gestes "inutiles" à l'acquisition que le calligraphe peut faire. En effet ce dernier peut par exemple vouloir étendre son bras pour se relaxer, geste qui sera pris en compte par le programme dans la reconstitution.

\section{Besoins fonctionnels et non fonctionnels}

	Une séparation des tâches était déjà effective dans le projet original, sur lequel travaillaient deux personnes : une personne s'occupait de l'acquisition stéréo, pendant que la seconde était sur la modélisation 3D de la plume. Cette séparation a été conservée dans ce TER : \textsc{Gallet} Benoît et \textsc{Herrmann} Emmanuel s'occupent de la première partie, tandis que \textsc{Charpignon} Thibault et \textsc{Réty} Martin ont pris la deuxième.

\subsection{Augmentation de la cadence d'acquisition}

Naturellement, différentes idées nous sont venues pour augmenter la cadence d'acquisition de la vidéo en stéréo. Nous les détaillons ici, même si par la suite cette liste sera sûrement étoffée. Le programme fonctionne suivant plusieurs étapes : Tout d'abord, une image est capturée à partir des deux caméras, puis les deux images sont encodées dans une vidéo. Une fois la capture finie, la vidéo est traitée afin de régler les problèmes de distorsion. Grâce à cette grande boucle qui capture les images deux par deux (une image par caméra), les vidéos finales commencent et terminent exactement au même moment, et permettent donc d'avoir exactement au même moment la feuille d'écriture filmée sous deux angles différents. La modélisation en 3D sous OpenGl est alors possible. La seule variable est que plus la cadence d'acquisition est élevée, plus il y aura de fps sur les vidéos finales, et plus la modélisation 3D de la plume sera précise. De plus, notre architecture et notre code doivent être assez robuste, pour que si un jour une troisième voire une quatrième caméra soient rajoutées, le nombre de fps ne redescende pas drastiquement.

\subsubsection{Programmation parallèle}

Grâce à la programmation parallèle, qu'elle soit au niveau du CPU avec de l'openmp ou des threads, ou au niveau du GPU avec CUDA, nous pensons pouvoir accélérer l'acquisition des images, et donc des fps sur les vidéos finales. Nous pensons regarder quelles parties peuvent être faites en parallèle, peut-être est-il possible d'uniquement récupérer une image tous les 3 centièmes de secondes (pour les 30 fps) dans le programme principal, et de faire tous les autres traitements dans des régions parallèles, avec par exemple un thread qui s'occupe d'ajouter la prochaine image à la vidéo, un autre thread qui enlève la distorsion de l'image, etc.

\subsubsection{Complexité}

Reprendre le code pour en examiner sa complexité est une autre piste envisagée pour augmenter les fps. Nous pensons séparer cette idée en deux étapes : Tout d'abord regarder la complexité de l'algorithme dans sa généralité, pour se rendre compte s'il y a problème ou non à ce niveau là, et voir les morceaux posant plus problème que le reste, puis faire des tests plus précisément sur ces parties pour voir précisément ce qui ne va pas. Une analyse légerement différente pourra être effectuée, avec des tests fontionnels calculant quelle partie prend le plus de temps. Ces tests sont très complémentaires des tests de complexité, à eux deux ils devraient mettre en exergue les problèmes principaux du code actuel.

\subsubsection{Modularité}

Outre cette analyse de la complexité, une mise au propre du code devra être effectuée. En effet, tout se trouve dans la fonction main, dans deux grandes boucles. Une partie de notre travail sera donc de modulariser cette fonction, de la séparer en plusieurs méthodes afin de gagner en clarté. La complexité ne devrait pas être touchée car nous rajoutons l'option \texttt{-O3} lors de la compilation afin d'avoir un code en une ligne. Ce nettoyage permettra de naviguer plus facilement dans le code, et de faciliter sa maintenance par la suite.

\subsubsection{Généralisation}

Sinon, le dernier axe sur lequel travailler sera la généralisation du nombre de caméra. Pour l'instant, tout dans le code est fait pour deux caméras, avec du code dupliqué deux fois pour chaque action. La généralisation pour n caméras sera facilitée par la modularisation du code, et permettra par la suite de rajouter une ou plusieurs caméras sans modification majeure du code, uniquement en changeant quelques \emph{\#define}.


\subsection{Tracking de la plume}

Comme pour la partie sur l'augmentation de la cadence d'acquisition, différents problèmes sont à résoudre pour le tracking. Cette partie permet de traiter les vid\'eos produites par les cam\'eras et d'en ressortir une trace des mouvements effectu\'es par le calligraphe. L'\'etudiant de Polytech qui a travaill\'e sur cette partie a recherch\'e diff\'erents algorithmes permettant d'effectuer ce tracking. Son \'etude se focalise sur deux algorithmes bas\'es sur l'apprentissage de toutes les apparences observ\'ees de l'objet et d'une estimation des erreurs pour ensuite les \'eviter:

\begin{itemize}

\item Tracking Learning Detection (TLD)

\item Kernelized Correlation Filters (KCF)

\end{itemize}

  
\subsubsection{Analyse de la complexité}

Heureusement, l'analyse des deux algorithmes a déjà été faite par l'\'etudiant, ce qui a montr\'e que dans notre cas l'algorithme KCF est le plus efficace. Son \'etude est bas\'ee sur plusieurs crit\`eres, la d\'eviation moyenne des deux vid\'eos, le nombre de frames et le temps de calcul. Seul le premier critère est r\'eellement diff\'erent entre les deux m\'ethodes. C'est cette diff\'erence qui a orient\'e son choix vers l'algortihme KCF. \\

Ici, notre premier axe de recherche sera orient\'e vers une \'etude compl\'ementaire de ces algorithmes pour v\'erifier la v\'eracit\'e de l'analyse pr\'ec\'edente. Pour cela nous allons r\'eutiliser les crit\`eres d'\'etudes et ensuite essayer d'en trouver d'autres pour confirmer le choix. Dans un second temps il nous faudra rechercher d'autres algorithmes ou m\'ethodes de programmation pour am\'eliorer le tracking.

\subsubsection{Gestion mouvements}

Bien entendu, le choix des algorithmes n'est pas la seule difficulté, nous faisons face \'egalement à des contraintes physiques li\'ees aux mouvements du calligraphe. Par exemple il doit prendre des temps de repos afin de garder sa fluidit\'e d'\'ecriture en faisant des gestes de relaxation du poignet. Ces mouvements ne doivent pas \^etre pris en compte par l'algorithme de tracking afin d'\'eviter des erreurs sur la repr\'esentation du mouvement. \\

Résoudre ce problème ce probl\`eme pourrait passer par la sauvegarde \`a un temps T et \`a un temps T+1 d'une image de la partie suivie. Puis analyser la diff\'erence entre les deux images et en ressortir un r\'esultat positif ou n\'egatif. Cela reviens \`a prendre la derni\`ere image o\`u le calligraphe \'ecrit et une autre image qui permettra de voir si le mouvement est la continuit\'e de l'\'ecriture ou un mouvement parasite.

\subsection{Autres axes de travail}

\subsubsection{Zone de capture}

En écrivant, le calligraphe doit de temps en temps bouger la feuille pour se repositionner et continuer sa rédaction. L'algorithme actuel ne gère pas ce mouvement, ce qui nécessitait après chaque mouvement de la feuille un nouveau calibrage des caméras et de la zone de capture. Une solution possible pour résoudre ce problème est la mise en place d'un système de cadre pour que le calligraphe sache la zone dans laquelle il peut écrire. Ce cadre pourrait être un marquage sur la feuille qui délimitera la zone de capture. Nous souhaitons également rechercher d'autre solutions possibles pour résoudre ce problème.

\subsubsection{Changement de la structure}

Pour le moment le dispositif de capture ne comporte que deux caméras et des angles de prises de vue bien définis. L'ajout d'une caméra et le repositionnement des deux premières peut permettre de rendre plus précise l'acquisition. De cette manière nous aurions à notre disposition des informations supplémentaires pour améliorer la reconstitution du mouvement. Il nous faut donc tester différentes configurations et choisir la meilleure. \\

Un des facteurs majeurs de la capture d'image est la lumière. En effet, il est important que la feuille soit bien éclairée pour le confort et l'écriture du calligraphe. La structure actuelle ne comporte pas d'éclairage du tout, il était nécessaire d'avoir une lampe d’appoint. Une solution simple est l'ajout d'un panneau LED pour avoir une luminosité uniforme sur toute la feuille. \\

Tous ces changements devront peut-être être accompagnés d'une refonte totale du dispositif.

\subsubsection{Compatibilité Windows - Linux - macOS}

A l'origine, les étudiants ont développé tout le code sur Windows et plus particulièrement sur l'IDE Visual Studio (C++). Pour rendre le code réutilisable à l'avenir nous avons comme objectif de pouvoir l'utiliser sur tous les systèmes d'exploitation (Linux/macOS en plus). Pour cela il est nécessaire d'uniformiser le code et de se servir de librairies communes pour standardiser au mieux le projet.

\section{Prototypes et résultats de tests préparatoires}

Pour bien prendre en main le projet il nous fallait tester réellement le dispositif de lancement du logiciel jusqu'à la capture vidéo. Nous avons du  procéder à l'installation de tout l'environnement de travail nécessaire (FlyCap 2, OpenCV, OpenGl) et l’acquisition des premières vidéos avec les caméras mises à notre disposition. \\

Le groupe affecté à la capture des vidéos en stéréo a pu transférer le code initial sous Linux et ainsi le tester directement. Les tests ont été concluants et ils ont pu commencer directement à améliorer le système. Le second groupe a quant à lui récupéré le code concernant le tracking mais a rencontré de gros problèmes lors de son passage de Windows vers Linux. \\
Le code n'utilisant pas des fonctions standards, important des librairies en "dur" et étant peu commenté, a rendu pour le moment impossible les tests du code de l'étudiant qui travaillait sur le tracking. Pour y remédier ils ont du repasser cette partie du projet sur Windows le temps de bien comprendre les différents problèmes et de les corriger.

\section{Architecture}

Le projet est scindé en deux grosse parties, presque indépendantes l'un de l'autre.

\begin{figure}[!htb]
\centering
\includegraphics[width=\textwidth]{Modules/Picture/Architecture.png}
\caption{architecture}
\label{architecture}
\end{figure}

Toute l'acquisition des vidéos se fait en parallèle dans la partie 

\section{Algorithmes et structures de données}

\subsection{Parallélisation OpenMP}
Afin de pouvoir faire de l'acquisition vidéo en simultané sur plusieurs caméras tout en gardant le plus possible d'images par seconde, nous avons opté pour une parallélisation de l'algorithme d'acquisition grâce à OpenMP. Ainsi nous affectons - dans la mesure du possible - une caméra à un thread en lançant la région parallèle de cette sorte :
\begin{verbatim}
#pragma omp parallel num_threads(numCameras)
\end{verbatim}
Avec \texttt{numCameras} le nombre de caméras détectées. \\
Juste avant de commencer la capture, nous posons une barrière à l'aide de 
\begin{verbatim}
#pragma omp barrier
\end{verbatim}
dans le but de déclencher la capture des caméras de la manière la plus synchronisée possible.
À l'intérieur de la boucle d'acquisition, une nouvelle barrière est mise avant chaque récupération du buffer de la caméra.
Dans le cas où l'utilisateur souhaite faire un affichage de la capture qu'il est en train d'effectuer, un des threads est choisi via
\begin{verbatim}
#pragma omp single
\end{verbatim}
afin de s'occuper de cet affichage.
Enfin, chaque thread écrit la frame dans la vidéo correspondante à sa caméra.

\subsection{Export des paramètres de la caméra}

La récupération des paramètres de la caméra, calculés via MatLab, va nous permettre de faire l'undistortion et la reconstruction 3D. Jusqu'à présent, ces paramètres étaient rentrés en dur dans les programmes, nous avons donc décidé de faire des imports/exports de ces données pour plus de simplicité et de ré\-u\-ti\-li\-sa\-bi\-li\-té.
Pour ce faire, une fois les paramètres de calibration calculés dans MatLab , on récupère un objet CameraParameters par caméra. Afin de générer les fichiers de configuration nécessaires, il suffit de rentrer deux commandes par caméra :

\begin{verbatim}
	dlmwrite( 													  \
	'*PATH_TO_ACQUISITION*/Calib_camera_*NUM_CAMERA*_Matlab.txt', \
	camera*NUM_CAMERA*.IntrinsicMatrix,'delimiter', ' ',		  \
	'precision', 5)
	dlmwrite(													  \
	'*PATH_TO_ACQUISITION*/Calib_camera_*NUM_CAMERA*_Matlab.txt', \
	horzcat(camera*NUM_CAMERA*.RadialDistorsion,				  \
	camera*NUM_CAMERA*.TangentialDistorsion),					  \
	'-append', 'delimiter', ' ', 'precision',5)
\end{verbatim}

Il faut remplacer $*NUM\_CAMERA*$ par le nom de l'objet de la caméra correspondante.

\subsection{Fonctionnement de la reconstruction 3D}

Pour faire la reconstruction nous avons gardé les différentes librairies utilisées dans le code fourni. Nous avons utilisé GLFW 3 comme libraire de gestion de fenêtre et de contexte OpenGL, GLM comme librairie mathématique permettant de manipuler des matrices de toute taille et GLEW comme librairie de gestion des extensions utilisées et surtout pour gérer la plateforme sur laquelle on utilise OpenGL. \\
 
La reconstruction 3D de ce qui est écrit dans la vidéo capturée se fait en fin de programme. Le tracking produit un fichier de points 3D qui sont utlisés pour la reconstruction avec OpenGL. Ce programme s'effectue en deux phases. \\

\subsubsection{Initialisation de la fenêtre et du contexte OpenGL}
La première phase consiste à initialiser les outils OpenGL utilisés. Dans un premier temps il a fallu initialiser la fenêtre OpenGL, c'est à dire attribuer la version d'OpenGL utilisée, une taille, et même lier les fonctions d'évenement OpenGL (mouvement de souris, touches clavier, etc) à la fenêtre. Ensuite il y a une initialisation du contexte, c'est à dire créer les outils qui permettent de lier les points 3D, la fenêtre et d'autres valeurs à la carte graphique (GPU). Pour cela on utilise des Vertex Array Object (VAO) qui permettent de lier à la carte graphique un programme nommé Shader, composé de FragmentShader qui sont utilisés pour dessiner dans la fenêtre. On utilise également des Vertex Buffer Object (VBO) qui nous permettent de lier des variables (ou structures) stockées dans le processeur (CPU) avec la GPU, qui permettront d'influer sur ce qui est dessiné dans la fenêtre par le Shader. \\

\subsubsection{Dessin des points dans le repère 3D}
La deuxième phase est la phase de traitement des données. Dans un premier temps il a fallu \textit{parser} le fichier dans lequel les points 3D sont contenus. Ensuite nous avons chargé dans deux VAO différents deux Shaders, un qui sera utilisé pour construire un plan dans le repère 3D et un second qui sera utilisé pour effectuer le dessin des points 3D du tracking. Pour le dessin des points lors du \textit{parsing} nous plaçons chaque point dans des matrices à quatre lignes et une colonne qui contiendront les coordonnées du repère x,y,z et w - w permettant de retrouver les coordonnés euclidienne du point. Une fois la liste remplie nous plaçons le point dans la fenêtre OpenGL à l'aide d'un savant calcul de placement de point qu'on nommera PVM. Ce PVM n'est rien d'autre qu'une suite de multiplications de matrices ordonnées : matrice de projection * matrice de vue * matrice de modèle qui permettent de rendre visible un dessin dans une fenêtre en fonction de la position de la camera et des coordonnées du point à dessiner.
La matrice du modèle (M) est obtenue par la translation T , la rotation R appliquée sur l'objet tel que : $R * T * v = M * v$ où v est un vecteur de l'objet.
La matrice de vue (V) est obtenue par la multiplication de M et un alignement des objets de la fenêtre par rapport à la vue humaine : $v' = V * M * v$ où v est un vecteur de l'objet.
La matrice de projection (P) est obtenue par la multiplication de V, M, et une projection dans la fenêtre des objets : $v' = P * V * M * v$ où v est un vecteur de l'objet.
Une fois ces valeurs de PVM envoyées à la GPU, le point est alors dessiné et est visible sur la fenêtre.
Pour le dessin du plan nous avons repris les paramètres PVM calculés precedement, puis effectué le dessin des points placés dans le repère.
Nous avons alors en visuel le plan et ce qui a été reconstruit en 3D, une représentation de l'écriture effectuée dans la vidéo. Il est alors possible de bouger dans la fenêtre avec des touches et d'orienter sa vue en fonction de la position de la souris.

\section{Complexité de l'acquisition}

\subsection{Acquisition}

\subsubsection{Acquisition stéréo et enregistrement des vidéos}

On assume le fait qu'il y ait n lignes et m colonnes dans une image. Nous proposons deux complexités, une borne min si on considère qu'un pixel de l'image se traite en temps $O(1)$, et une borne max si l'on considère que le pixel se traite en temps $O(3)$ à cause de ses caractéristiques RGB.

\textbf{Borne min}

Récupération de l'image brute (RetrieveBuffer) : $O(n*m)$

Conversion de l'image brute vers RGB (rawImage.Convert) : $O(n*m)$

Création de la matrice de l'image RGB (imageRGB=cv::Mat) : $O(n*m)$

Ecriture d'une image dans la vidéo (outputVideo.write) : $O(n*m)$

Ce qui nous donne un total de temps d'exécution en $O(4(n*m))$.

\textbf{Borne max}

Il suffit de multiplier par trois les valeurs précédentes, ce qui nous donne au total $O(12(n*m))$

\textbf{Analyse}

Nous pouvons donc en conclure que le temps d'exécution de l'algorithme C est :
$O(4(n*m)) \leq C \leq O(12(n*m))$

C'est-à-dire qu'il s'exécute globalement en temps linéaire en la taille de l'entrée. Cela dit, il faut l'exécuter à chaque tour de boucle (à chaque acquisition d'image), donc on pourrait dire que sa complexité serait alors de $O(n*m*f)$, f étant le nombre de frames que nous enregistrons. Evidemment, cela vaut pour la version parallèle du code, si on prend en compte la version séquentielle, il faut bien sûr multiplier cette complexité par le nombre de caméras que l'on a.

\section{Tests de fonctionnement et de validation}

Différents tests de validation ont été effectués tout au long du projet afin de s'assurer du bon fonctionnement de l'application. Ces tests pourront être refaits par la suite en décommentant les $\sharp$ define nécessaires, et pourront donc être réutilisés par des étudiants reprenant ce projet.

\subsection{Latence}

La latence est une caractéristique incontournable de notre projet, le client nous ayant précisé plusieurs fois que c'était très important qu'il y ait le moins de décalage possible entre les deux vidéos. Plus le décalage est grand et moins la reconstruction en 3D par la suite sera précise. Nous avons donc mesuré précisément le temps qui sépare deux frames dans le code. 

\begin{figure}[!h]
\centering
\includegraphics[width=\textwidth, height=10cm]{Modules/Picture/latence.png}
\caption{Latence}
\label{latence}
\end{figure}

Ce diagramme en bâton (Figure \ref{latence}) montre cette latence. Elle est plus importante avec le parallélisme qu'avec le programme séquentiel. Il faudra donc que le client pèse le pour et le contre sur la méthode qu'il préfère utiliser. Avec le parallelisme, le nombre de caméra pourra être augmenté (de façon raisonnable) sans chute de fps, car chaque coeur de la machine s'occupera d'une caméra particulière, par contre, la latence est plus importante(5 à 6 ms) . Avec la version séquentielle du code, la latence est plus que minime (1/2 millième de seconde environ), mais l'ajout d'une à plusieurs caméras fera chuter le nombre de fps.
De plus, la latence sera surement plus constante dans la région parallèle, car chaque thread s'occupant d'une caméra, celle-ci n'augmentera pas si on ajoute plus de caméras (dans la limite des coeurs disponibles sur la machine). Par contre, séquentiellement, comme on lance les acquisitions les unes après les autres, plus il y a de caméras, plus celles qui sont lancées en dernier auront de décalage avec la première.

\subsection{FPS}

\section{Bilan}

Nous avions différents objectifs au départ, concernant les deux parties de notre TER, acquisition et tracking/reconstruction.

Le premier était d'avoir une vidéo à 30 fps. Précedemment, le programme capturait les images à la vitesse de 7 ou 8 images/secondes, et actuellement on crée des vidéos en 30 fps constants, et en plus avec de la parallélisation, ce qui nous laisse à penser que le nombre de caméras peut augmenter sans baisser ce chiffre.

Il fallait ensuite généraliser le code de l'acquisition pour qu'il puisse marcher pour n caméras. Au départ, tout était fait pour deux caméras, avec du code copié-collé en deux fois pour chacune des caméras. Tout cela a été modifié, avec des tableaux de taille n (pour n caméras). Par contre, pour la partie reconstruction 3D, cela marche toujours avec deux caméras. Pour modifier cela, il faudrait changer la méthode de reconstruction 3D de points qui actuellement ne prend les points que de deux caméras. Une autre solution serait de lancer le programme sur toutes les paires de caméras et de faire ensuite la moyenne des points récoltés pour plus de réalisme dans la reconstruction.

Le code devait être compilable et exécutable sous environnement UNIX, les étudiants précédents l'ayant fait marcher uniquement sous Windows. Ce point a été également réalisé, une partie de ce rapport (section Manuel) a même été fait pour permettre aux prochains étudiants qui travailleront sur ce projet d'installer les frameworks et bibliothèques nécessaires à la compilation du code, ainsi que des explications sur l'utilisation de l'application.

Pouvoir bouger la feuille n'a malheureusement pas pu être fait par manque de temps, cependant nous avons réfléchi à une solution pour le faire, qui est détaillée dans la partie Extensions et améliorations possibles.

Il fallait rendre le code plus lisible. Ceci a également été fait, grâce à un gros travail de documentation et de compréhension du code. En effet, la majeure partie n'était pas commentée, avec des noms de variables non explicites, etc. Cela a été amélioré, avec une documentation Doxygen, afin que le code puisse être compris par tous.

Enfin, en naviguant sur la documentation de PointGrey (le fabricant des caméras), nous nous sommes rendu compte qu'il existait déjà un moyen de faire de l'acquisition stéréo, grâce à des fonctions dans l'API fournie, et à un cable reliant les caméras. Peut-être faudrait-il envisager aussi cette solution dans le futur, même si notre travail d'amélioration du code de l'étudiant précédent s'étant occupé de cette partie est désormais finie.

\section{Extensions et améliorations possibles}

\subsection{Droits d'auteur}
Au fur et à mesure que nous essayions de comprendre le code de la reconstruction 3D (ce qui a été long car il a été fourni sans doc, commentaires et explications d'utilisation, avec une syntaxe et des imports pour Windows), nous nous sommes rendu compte que sur cette partie plusieurs morceaux de code avaient été copiés sur Internet.

La classe Shader, ainsi que la moitié de la classe Camera provient de \url{http://blog.csdn.net/sinat_26989191/article/details/51205149}, un outil pour faire un système solaire en 3D.

La classe HOG est presque entièrement copié collée de la fin de ce programme ( \url{http://lib.csdn.net/snippet/cplusplus/28974} ), qui est sous copyright.

Diverses autres parties du code proviennent de copiés-collés de différentes documentations ou exemples, ce qui est dans une certaine mesure moins dé\-ran\-geant.

Il y a certainement d'autres endroits du projet qui proviennent d'Internet, tout n'a pas été vérifié par manque de temps. Au total, nous estimons qu'au moins 50\% du code n'a pas été écrit par le groupe précédent. Cela implique que si ce projet est repris par un groupe futur, il faudra réécrire ces parties du code, ou du moins indiquer clairement les sources et les différents copyrights sur les classes copiées.

\subsection{Interface graphique}

Une extension possible serait de réaliser une interface graphique, par exemple en Qt. Cela permettrait un maniement beaucoup plus facile de l'application, et rendrait possible son utilisation par des personnes extérieures. En effet, en ce moment tout s'exécute dans le terminal, ce qui peut être moins intuitif. Avec cette interface graphique, on pourrait par exemple choisir de sélectionner une caméra pour voir ce qu'elle enregistre, définir différentes options comme choisir l'algorithme de tracking à utiliser, les sorties vidéos ou images à effectuer, ...

\section{Planning}

Les petites barres sur les diagrammes de Gantt (Figure \ref{gantt} et suivantes) correspondent aux différentes réunions que l'on a eu, suivies des noms des participants. Les plus grandes correspondent quant à elles aux tâches effectuées. Il y en a peu pour le moment car la phase de compréhension et d'installation des composants a été relativement longue.

\newpage
\begin{figure}
	\begin{center}
		\includegraphics[scale=0.3]{Modules/Picture/gantt_0_1}
		\caption{Diagramme de Gantt 1/2}
		\label{gantt}
	\end{center}
\end{figure}

\newpage
\begin{figure}
	\begin{center}
		\includegraphics[scale=0.3, angle=90]{Modules/Picture/gantt_final_2}
		\caption{Diagramme de Gantt 2/2}
	\end{center}
\end{figure}

\newpage

\begin{figure}[!h]
\centering
\includegraphics[scale=0.35, angle=90]{Modules/Picture/tableau_gantt_final}
\caption{Diagramme de Gantt - Total}
\label{ganttTableau}
\end{figure}

\clearpage

\newpage

\section{Manuel d'utilisation}

Pour pouvoir lancer et utiliser cette application, différentes bibliothèques doivent être installées sur l'ordinateur. Toutes les manipulations sont décrites pour un environnement Linux (ou Mac), mais peuvent être transposées assez facilement pour Windows.

\subsection{Installation}

\subsubsection{FlyCap}

FlyCap est le framework de PointGrey, le fabricant des deux caméras, permettant de communiquer dans un programme avec les caméras. L'installation est relativement aisée, il suffit d'aller sur le site du constructeur (https://www.ptgrey.com/support/downloads) pour télécharger le framework. Il faut alors renseigner le type de caméra (Chameleon3), le modèle (CM3-U3-13S2C-CS) puis le système d'exploitation utilisé. Il faut ensuite sélectionner "Latest FlyCapture2 Full SDK" dans la partie software. Une fois le téléchargement terminé, un README détaille toutes les étapes nécessaires à l'installation, il suffit de les suivre une à une pour avoir FlyCap en état de marche. Pour vérifier si le logiciel s'est bien installé, il est possible de le lancer dans un terminal via la commande \textit{flycap} pour voir la vue d'une des caméras.

\subsubsection{OpenCV}

OpenCV doit être installé également, afin de profiter de son traitement d'image et de ses algorithmes de tracking. Il a été décidé d'utiliser OpenCV 3, car celui-ci contient un module optionnel OpenCV_contrib. Ce dernier contient tous les algorithmes de tracking, c'est un extra-module indispensable pour pouvoir exécuter le projet.

Pour installer OpenCV, voici la marche à suivre :

- On se place dans le dossier de travail et on télécharge les dernières versions OpenCV  et d'OpenCV_contrib.

cd ~/<my_working_directory>
git clone https://github.com/opencv/opencv.git
git clone https://github.com/opencv/opencv_contrib.git

- On se place dans le dossier OpenCV téléchargé et on crée un répertoire temporaire pour le build.

cd ~/opencv
mkdir build
cd build

- On lance la configuration en n'oubliant pas d'inclure l'extra-module. Cette étape peut prendre un certain temps, jusqu'à 1h30 suivant les machines.

cmake -D CMAKE_BUILD_TYPE=Release -DOPENCV_EXTRA_MODULES_PATH=*PATH_TO_FOLDER*/opencv_contrib/modules -D CMAKE_INSTALL_PREFIX=/usr/local ..

- On démarre l'installation. Si on ne possède que quatre coeur, mieux vaut marquer -j4.

make -j7

- Dans certains cas, il faut rajouter la ligne suivante dans le .bashrc pour indiquer où se trouvent les librairies OpenCV.

export LD_LIBRARY_PATH=\$LD_LIBRARY_PATH:/usr/local/lib



\subsubsection{OpenGL, glew, GLFW et glm}

- Il faut commencer par installer OpenGL et Mesa :

sudo apt-get install cmake xorg-dev libglu1-mesa-dev

- Pour savoir si l'opération s'est bien passée, il faut regarder que les deux fichiers suivants sont bien présents :

/usr/include/GL
/usr/lib/x86_64-linux-gnu/libGL.so

- A présent, il faut retourner dans le dossier de travail pour installer GLFW. Commencer par télécharger le code source (https://sourceforge.net/projects/glfw/files/glfw/3.0.4/glfw-3.0.4.zip/download), puis exécuter les instructions suivantes :

cd glfw-3.0.4
rehash
cmake -G "Unix Makefiles"
make
sudo make install

- Les fichiers suivants doivent maintenant être présents :

/usr/local/include/GLFW
/usr/local/lib/libglfw3.a

\subsection{Utilisation}

Le travail est séparé en deux parties.

g++ -std=c++11 -lglfw -O2 -Wall -o test main.cpp `pkg-config --cflags --libs opencv` 

\newpage

\bibliographystyle{plain}
\nocite{*}
\bibliography{biblio}

 
\end{document}

\grid
